\documentclass[10pt, letterpaper]{article}

% Packages:
\usepackage[
    ignoreheadfoot, % set margins without considering header and footer
    top=2 cm, % seperation between body and page edge from the top
    bottom=2 cm, % seperation between body and page edge from the bottom
    left=2 cm, % seperation between body and page edge from the left
    right=2 cm, % seperation between body and page edge from the right
    footskip=1.0 cm, % seperation between body and footer
% showframe % for debugging
]{geometry} % for adjusting page geometry
\usepackage[explicit]{titlesec} % for customizing section titles
\usepackage{tabularx} % for making tables with fixed width columns
\usepackage{array} % tabularx requires this
\usepackage[dvipsnames]{xcolor} % for coloring text
\definecolor{primaryColor}{RGB}{0, 79, 144} % define primary color
\usepackage{enumitem} % for customizing lists
\usepackage{fontawesome5} % for using icons
\usepackage{amsmath} % for math
\usepackage[
    pdftitle={John Doe's CV},
    pdfauthor={John Doe},
    colorlinks=true,
    urlcolor=primaryColor
]{hyperref} % for links, metadata and bookmarks
\usepackage[pscoord]{eso-pic} % for floating text on the page
\usepackage{calc} % for calculating lengths
\usepackage{bookmark} % for bookmarks
\usepackage{lastpage} % for getting the total number of pages
\usepackage{changepage} % for one column entries (adjustwidth environment)
\usepackage{paracol} % for two and three column entries
\usepackage{ifthen} % for conditional statements
\usepackage{needspace} % for avoiding page brake right after the section title
\usepackage{iftex} % check if engine is pdflatex, xetex or luatex

% 한국어 폰트를 위해
%\usepackage{kotex}
\usepackage{luatexko}
\usepackage{fontspec}
\setmainfont{Nanum Myeongjo} % 나눔고딕 폰트 사용
\setsansfont{Nanum Myeongjo}

% Ensure that generate pdf is machine readable/ATS parsable:
\ifPDFTeX
\input{glyphtounicode}
\pdfgentounicode=1
\usepackage[T1]{fontenc}
\usepackage[utf8]{inputenc}
\usepackage{lmodern}
\fi

\usepackage[default, type1]{sourcesanspro}

% Some settings:
\AtBeginEnvironment{adjustwidth}{\partopsep0pt} % remove space before adjustwidth environment
\pagestyle{empty} % no header or footer
\setcounter{secnumdepth}{0} % no section numbering
\setlength{\parindent}{0pt} % no indentation
\setlength{\topskip}{0pt} % no top skip
\setlength{\columnsep}{0.15cm} % set column seperation
\makeatletter
\let\ps@customFooterStyle\ps@plain % Copy the plain style to customFooterStyle
\patchcmd{\ps@customFooterStyle}{\thepage}{
    \color{gray}\textit{\small 손현준 - Page \thepage{} of \pageref*{LastPage}}
}{}{} % replace number by desired string
\makeatother
\pagestyle{customFooterStyle}

\titleformat{\section}{
% avoid page braking right after the section title
    \needspace{4\baselineskip}
    % make the font size of the section title large and color it with the primary color
    \Large\color{primaryColor}
}{
}{
}{
% print bold title, give 0.15 cm space and draw a line of 0.8 pt thickness
% from the end of the title to the end of the body
    \textbf{#1}\hspace{0.15cm}\titlerule[0.8pt]\hspace{-0.1cm}
}[] % section title formatting

\titlespacing{\section}{
% left space:
    -1pt
}{
% top space:
    0.3 cm
}{
% bottom space:
    0.2 cm
} % section title spacing

% \renewcommand\labelitemi{$\vcenter{\hbox{\small$\bullet$}}$} % custom bullet points
\newenvironment{highlights}{
    \begin{itemize}[
        topsep=0.10 cm,
        parsep=0.10 cm,
        partopsep=0pt,
        itemsep=0pt,
        leftmargin=0.4 cm + 10pt
    ]
    }{
    \end{itemize}
} % new environment for highlights

\newenvironment{highlightsforbulletentries}{
    \begin{itemize}[
        topsep=0.10 cm,
        parsep=0.10 cm,
        partopsep=0pt,
        itemsep=0pt,
        leftmargin=10pt
    ]
    }{
    \end{itemize}
} % new environment for highlights for bullet entries


\newenvironment{onecolentry}{
    \begin{adjustwidth}{
        0.2 cm + 0.00001 cm
    }{
        0.2 cm + 0.00001 cm
    }
    }{
    \end{adjustwidth}
} % new environment for one column entries

\newenvironment{twocolentry}[2][]{
    \onecolentry
    \def\secondColumn{#2}
    \setcolumnwidth{\fill, 4.5 cm}
    \begin{paracol}{2}
    }{
        \switchcolumn \raggedleft \secondColumn
    \end{paracol}
    \endonecolentry
} % new environment for two column entries

\newenvironment{threecolentry}[3][]{
    \onecolentry
    \def\thirdColumn{#3}
    \setcolumnwidth{1 cm, \fill, 4.5 cm}
    \begin{paracol}{3}
    {\raggedright #2} \switchcolumn
    }{
    \switchcolumn \raggedleft \thirdColumn
    \end{paracol}
    \endonecolentry
} % new environment for three column entries

\newenvironment{header}{
    \setlength{\topsep}{0pt}\par\kern\topsep\centering\color{primaryColor}\linespread{1.5}
    }{
    \par\kern\topsep
} % new environment for the header

% save the original href command in a new command:
\let\hrefWithoutArrow\href

% new command for external links:
\renewcommand{\href}[2]{\hrefWithoutArrow{#1}{\mbox{\ifthenelse{\equal{#2}{}}{ }{#2 }\raisebox{.15ex}{\footnotesize \faExternalLink*}}}}


\begin{document}
    \placelastupdatedtext
    \begin{header}
        \fontsize{15 pt}{15 pt}
        \textbf{손현준}

        \vspace{0.3 cm}

        \normalsize
        \mbox{\hrefWithoutArrow{mailto:guswns1659@gmail.com}{{\footnotesize\faEnvelope[regular]}\hspace*{0.13cm}guswns1659@gmail.com}}
        \kern 0.5 cm
        \mbox{\hrefWithoutArrow{tel:010-7720-7957}{{\footnotesize\faPhone*}\hspace*{0.13cm}010-7720-7957}}
        \kern 0.5 cm
        \mbox{\hrefWithoutArrow{https://guswns1659.github.io}{{\footnotesize\faLink}\hspace*{0.13cm}https://guswns1659.github.io}}
        \kern 0.5 cm
        \mbox{\hrefWithoutArrow{https://github.com/guswns1659}{{\footnotesize\faGithub}\hspace*{0.13cm}https://github.com/guswns1659}}
        \kern 0.5 cm
    \end{header}

    \vspace{0.3 cm - 0.3 cm}


    \section{Who AM I}

    \begin{onecolentry}
        \begin{highlightsforbulletentries}


            \item 개발자는 필요하면 만들고, 누군가의 불편함을 해결해줄 수 있어야 한다고 생각합니다. 이론만 앞서지 않고 현실 속 문제를 해결하기 위해 노력합니다.

            \item 계획을 세우고 피드백을 통한 성장을 좋아합니다. 스프린트 방식의 플래닝과 회고를 통해 점진적 발전을 추구합니다.

        \end{highlightsforbulletentries}
    \end{onecolentry}

    \section{Why Developer}

    \begin{onecolentry}
        \begin{highlightsforbulletentries}

            \item 개발자는 세상에 건강한 영향력을 줄 수 있는 직업이라 생각합니다. 기술로 시각장애인과 자원봉사자를 연결해주는 앱 \href{https://www.bemyeyes.com}{Be My Eyes}에 자원봉사자로 참여하며 느꼈습니다.

            \item 성장을 측정할 수 있는 직업이기 때문입니다. 역량이 성장한 것을 눈에 보이는 결과물로 측정하여 점진적인 발전을 할 수 있다는 점이 매력적입니다

        \end{highlightsforbulletentries}
    \end{onecolentry}

    \section{Experience}


    \begin{twocolentry}{
        2023.12 - 2024.02

        2개월
    }
        \textbf{계좌 연결 시 1원인증 스킵 프로젝트}
        \begin{highlights}

            \item \textbf{설명} : 10대와 50대에서 높은 이탈률을 보이는 1원인증 단계를 조건에 따라 생략해 사용자 경험 증대
            \item \textbf{성과} : 1원인증 스킵 기술 적용해 계좌 연결 전환율 8\% 증가 (26.1\%->33.7\%)
            \item \textbf{역할} :
            \begin{highlights}
                \item webFlux이용해 short-circuit evaluation 적용. 동기호출 대비 레이턴시 2초 -> 0.5초 개선.
            \end{highlights}
            \item \textbf{기술} : WebFlux
        \end{highlights}
    \end{twocolentry}

    \vspace{0.2 cm}

    \begin{twocolentry}{
        2024.02 - 2024.08

        6개월
    }
        \textbf{실시간 택스 리펀 프로젝트}
        \begin{highlights}

            \item \textbf{설명} : 카카오페이 통해서 즉시 부가세를 환급받아 사용자 편의성 증대
            \item \textbf{역할} :
            \begin{highlights}
                \item UTC+1 차이를 고려한 안정적인 대사 및 정산 설계 및 개발 진행
                \item 중복 호출 시에도 정상적인 동작을 위한 멱등적 API 설계
                \item 해외 알리페이 개발 담당자와 API 및 네트워크 통신을 위한 커뮤니케이션 담당
            \end{highlights}
            \item \textbf{기술} : Spring batch
        \end{highlights}
    \end{twocolentry}

    \vspace{0.2 cm}

    \begin{twocolentry}{
        2023.07 - 2024.01

        5개월
    }
        \textbf{쓸수록 모이는 소비적금 프로젝트}
        \begin{highlights}

            \item \textbf{설명} : 결제한 금액의 일부를 적금으로 모아주는 편리한 기능 제공해 수익성 증대
            \item \textbf{성과} : 적금 계좌 20,000건 개설
            \item \textbf{역할} :
            \begin{highlights}
                \item 송금 금액 정합성을 위한 대사 및 정산 로직 개발
                \item 은행 지연 대비 서킷브레이커 적용
                \item 여러 프로젝트에서 반복되어 사용되는 송금 기능을 모듈화하여 중복 제거
                \item 스프린트 및 이벤트 스토밍 진행하여 PM, 개발자 동일한 컨텍스트 공유. 정확한 리소스 파악
            \end{highlights}
            \item \textbf{기술} : 스프링 서킷브레이커, 레디스, 카프카, 스프링 배치
        \end{highlights}
    \end{twocolentry}

    \vspace{0.2 cm}

    \begin{twocolentry}{
        2022.10 - 2023.01

        3개월
    }
        \textbf{은행 계좌 다건을 페이에 한번에 연결하는 프로젝트}
        \begin{highlights}

            \item \textbf{설명} : 기존 단건 계좌 연결을 개선해 평균 5건 이상 은행 계좌 한번에 연결해 사용자 경험 증대 프로젝트
            \item \textbf{성과} : 월 평균 계좌 연결 건수 85\% 증가 (54만건 -> 100만건)
            \item \textbf{역할} :
            \begin{highlights}
                \item webflux 병렬성 이용하여 일반 동기호출 대비 레이턴시 3초 -> 0.5초로 개선. (평균 10건 기준)
                \item 타임아웃 처리를 위한 이벤트 발행 및 폴링 방식 도입
            \end{highlights}
            \item \textbf{기술} : WebFlux, kafka
        \end{highlights}
    \end{twocolentry}

    \vspace{0.2 cm}

    \begin{twocolentry}{
        2022.07 - 2022.11

        4개월
    }
        \textbf{마이데이터 계좌 송금 프로젝트}
        \begin{highlights}

            \item \textbf{설명} : 연결된 마이데이터 계좌에서 타 계좌로 송금할 수 있는 기능 제공해 트래픽 증대에 기여한 프로젝트
            \item \textbf{성과} : 분기 대비 송금 일 건수 28\% 증가 (180만 건 -> 230만건)
            \item \textbf{역할} :
            \begin{highlights}
                \item 잔액이 변동되기에 정합성이 중요한 기능이라 레디스를 이용한 분산락을 사용
                \item 타임아웃 등 데이터 정합성 불일치 건들 재처리하는 배치 개발
                \item 코드리뷰 비용을 줄이고 개발 팀원 모두가 동일한 정보를 공유하기 위해 페어프로그래밍 진행
            \end{highlights}
            \item \textbf{기술} : 레디스
        \end{highlights}
    \end{twocolentry}

    \vspace{0.2 cm}

    \begin{twocolentry}{
        2022.03 - 2022.06

        4개월
    }
        \textbf{주식 미수금액용 충전 프로젝트}
        \begin{highlights}

            \item \textbf{설명} : 평균 초당 13건 발생하는 미수금을 위한 충전 기능 제공하여 주식 사용자 편리성 강화
            \item \textbf{성과} : 잔액 정합성 유지하며 45,000건/시간, 13TPS 트래픽 처리, 약 13억/일 충전금액 증대 (peak 기준. 10건당 30만원 책정)
            \item \textbf{역할} :
            \begin{highlights}
                \item 잔액이 변동되기에 정합성이 중요한 기능이라 레디스를 이용한 분산락을 사용
                \item ngrinder 이용한 성능 테스트를 통해 적정 TPS 확인
            \end{highlights}
            \item \textbf{기술} : 레디스, ngrinder
        \end{highlights}
    \end{twocolentry}

    \vspace{0.2 cm}


    \section{Trouble Shooting}




    \begin{highlights}

        % TODO : uri에 한글있으면 링크 실패
        \item \href{https://guswns1659.github.io/posts/}{HibernateCursorItemReader로 대용량 데이터 처리 시 OOME 발생 이유와 해결방법}
        \item \href{https://guswns1659.github.io/posts/}{스프링부트 부팅 시 성능 저하와 Warm-up}
        \item \href{https://guswns1659.github.io/posts/}{Spring WebClient 사용 시 uri tag가 none으로 찍히는 이슈}

    \end{highlights}

    \vspace{0.2 cm}


    \section{Side Projects}


    \begin{highlights}

        \item 이벤트 스토밍을 통한 PM, 개발자 컨텍스트 공유, 2주 단위 스프린트 도입하여 점진적 발전 기여
        \item 코드리뷰 지연 방지를 위한 PR 알림 봇
        \begin{highlights}
            \item MSA로 프로젝트가 많아지면서 혼자 프로젝트를 진행하는 경우 리뷰가 늦어지는 경우 발생
            \item 매일 아침 approved 받지 못해 리뷰가 필요한 pr 모아서 알림
            \item 결과 : 평균 00건 -> 00건
        \end{highlights}
        \item 배포 브랜치 자동 관리 시스템
        \begin{highlights}
            \item 여러 기능이 동시에 진행될 때 베타 테스트의 형상이 달라지는 경우가 발생
            \item 배포 브랜치를 사람이 관리하지 않고 github hook 이벤트 이용해 자동 관리하도록 변경
        \end{highlights}

    \end{highlights}



    \section{Education}




    \begin{twocolentry}{
        2012 - 2019
    }
        \begin{highlights}

            \item \textbf{경희대학교}, 호텔경영학과

        \end{highlights}
    \end{twocolentry}

    \vspace{0.2 cm}

    \begin{twocolentry}{
        2020
    }
        \begin{highlights}

            \item \textbf{코드스쿼드 부트캠프}, 백엔드 과정 수료

        \end{highlights}
    \end{twocolentry}


    \section{Specical Experience \&& Certificate}


    \begin{twocolentry}{
        2024
    }
        \begin{highlights}

            \item SQLD

        \end{highlights}
    \end{twocolentry}

    \vspace{0.2 cm}


    \begin{twocolentry}{
        2022
    }
        \begin{highlights}

            \item \textbf{코드스쿼드 리뷰어}, 수강생 프로젝트 코드리뷰 지원

        \end{highlights}
    \end{twocolentry}

    \vspace{0.2 cm}

    \begin{twocolentry}{
        2020
    }
        \begin{highlights}

            \item \href{https://launchpad.net/~ubuntu-l10n-ko}{우분투 한국어 번역 오픈소스 컨트리뷰톤 참여}

        \end{highlights}
    \end{twocolentry}



    \section{Technologies}


    \begin{highlights}
        \item \textbf{Languages:} Java, Kotlin
    \end{highlights}

    \vspace{0.2 cm}

    \begin{highlights}
        \item \textbf{Framework \&& Library:} Spring, JPA, Gradle
    \end{highlights}

    \vspace{0.2 cm}

    \begin{highlights}
        \item \textbf{Infra: } Kubernetes, Mysql, Redis, Kafka
    \end{highlights}


\end{document}