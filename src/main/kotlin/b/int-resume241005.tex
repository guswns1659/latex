\documentclass[10pt, letterpaper]{article}

% Packages:
\usepackage[
    ignoreheadfoot, % set margins without considering header and footer
    top=2 cm, % seperation between body and page edge from the top
    bottom=2 cm, % seperation between body and page edge from the bottom
    left=2 cm, % seperation between body and page edge from the left
    right=2 cm, % seperation between body and page edge from the right
    footskip=1.0 cm, % seperation between body and footer
% showframe % for debugging
]{geometry} % for adjusting page geometry
\usepackage[explicit]{titlesec} % for customizing section titles
\usepackage{tabularx} % for making tables with fixed width columns
\usepackage{array} % tabularx requires this
\usepackage[dvipsnames]{xcolor} % for coloring text
\definecolor{primaryColor}{RGB}{0, 79, 144} % define primary color
\usepackage{enumitem} % for customizing lists
\usepackage{fontawesome5} % for using icons
\usepackage{amsmath} % for math
\usepackage[
    pdftitle={HyunJun Son's CV},
    pdfauthor={HyunJun Son},
    colorlinks=true,
    urlcolor=primaryColor
]{hyperref} % for links, metadata and bookmarks
\usepackage[pscoord]{eso-pic} % for floating text on the page
\usepackage{calc} % for calculating lengths
\usepackage{bookmark} % for bookmarks
%\usepackage{lastpage} % for getting the total number of pages
\usepackage{changepage} % for one column entries (adjustwidth environment)
\usepackage{paracol} % for two and three column entries
\usepackage{ifthen} % for conditional statements
\usepackage{needspace} % for avoiding page brake right after the section title
\usepackage{iftex} % check if engine is pdflatex, xetex or luatex

% 한국어 폰트를 위해
%\usepackage{kotex}
\usepackage{luatexko}
\usepackage{fontspec}
\setmainfont{Nanum Myeongjo} % 나눔고딕 폰트 사용
\setsansfont{Nanum Myeongjo}

% 기본 텍스트 크기를 위해
\renewcommand{\normalsize}{\small}

% Ensure that generate pdf is machine readable/ATS parsable:
\ifPDFTeX
\input{glyphtounicode}
\pdfgentounicode=1
\usepackage[T1]{fontenc}
\usepackage[utf8]{inputenc}
\usepackage{lmodern}
\fi

\usepackage[default, type1]{sourcesanspro}

% Some settings:
\AtBeginEnvironment{adjustwidth}{\partopsep0pt} % remove space before adjustwidth environment
\pagestyle{empty} % no header or footer
\setcounter{secnumdepth}{0} % no section numbering
\setlength{\parindent}{0pt} % no indentation
\setlength{\topskip}{0pt} % no top skip
\setlength{\columnsep}{0.15cm} % set column seperation
\makeatletter
\let\ps@customFooterStyle\ps@plain % Copy the plain style to customFooterStyle
%\patchcmd{\ps@customFooterStyle}{\thepage}{
%    \color{gray}\textit{\small 손현준 - Page \thepage{} of \pageref*{LastPage}}
%}{}{} % replace number by desired string
\makeatother
\pagestyle{customFooterStyle}

\titleformat{\section}{
% avoid page braking right after the section title
    \needspace{4\baselineskip}
    % make the font size of the section title large and color it with the primary color
    \Large\color{primaryColor}
}{
}{
}{
% print bold title, give 0.15 cm space and draw a line of 0.8 pt thickness
% from the end of the title to the end of the body
    \textbf{#1}\hspace{0.15cm}\titlerule[0.8pt]\hspace{-0.1cm}
}[] % section title formatting

\titlespacing{\section}{
% left space:
    -1pt
}{
% top space:
    0.3 cm
}{
% bottom space:
    0.2 cm
} % section title spacing

% \renewcommand\labelitemi{$\vcenter{\hbox{\small$\bullet$}}$} % custom bullet points
\newenvironment{highlights}{
    \begin{itemize}[
        topsep=0.10 cm,
        parsep=0.10 cm,
        partopsep=0pt,
        itemsep=0pt,
        leftmargin=0.4 cm + 10pt
    ]
    }{
    \end{itemize}
} % new environment for highlights

\newenvironment{highlightsforbulletentries}{
    \begin{itemize}[
        topsep=0.10 cm,
        parsep=0.10 cm,
        partopsep=0pt,
        itemsep=0pt,
        leftmargin=10pt
    ]
    }{
    \end{itemize}
} % new environment for highlights for bullet entries


\newenvironment{onecolentry}{
    \begin{adjustwidth}{
        0.2 cm + 0.00001 cm
    }{
        0.2 cm + 0.00001 cm
    }
    }{
    \end{adjustwidth}
} % new environment for one column entries

\newenvironment{twocolentry}[2][]{
    \onecolentry
    \def\secondColumn{#2}
    \setcolumnwidth{\fill, 4.5 cm}
    \begin{paracol}{2}
    }{
        \switchcolumn \raggedleft \secondColumn
    \end{paracol}
    \endonecolentry
} % new environment for two column entries

\newenvironment{threecolentry}[3][]{
    \onecolentry
    \def\thirdColumn{#3}
    \setcolumnwidth{1 cm, \fill, 4.5 cm}
    \begin{paracol}{3}
    {\raggedright #2}
        \switchcolumn
        }{
        \switchcolumn \raggedleft \thirdColumn
    \end{paracol}
    \endonecolentry
} % new environment for three column entries

\newenvironment{header}{
    \setlength{\topsep}{0pt}\par\kern\topsep\centering\color{primaryColor}\linespread{1.5}
    }{
    \par\kern\topsep
} % new environment for the header

% save the original href command in a new command:
\let\hrefWithoutArrow\href

% new command for external links:
\renewcommand{\href}[2]{\hrefWithoutArrow{#1}{\mbox{\ifthenelse{\equal{#2}{}}{ }{#2 }\raisebox{.15ex}{\footnotesize \faExternalLink*}}}}

% TODO 추가해볼만한 거 : 알리랑 했던 rsa 암호화, 계좌번호 등 개인정보 aes 암호화, sFTP 파일 서버 접속 및 관리

\begin{document}
%    \placelastupdatedtext
    \begin{header}
        \fontsize{15 pt}{15 pt}
        \textbf{손현준}

        \vspace{0.3 cm}

        \normalsize
        \mbox{\hrefWithoutArrow{mailto:guswns1659@gmail.com}{{\footnotesize\faEnvelope[regular]}\hspace*{0.13cm}guswns1659@gmail.com}}
        \kern 0.5 cm
        \mbox{\hrefWithoutArrow{tel:010-7720-7957}{{\footnotesize\faPhone*}\hspace*{0.13cm}010-7720-7957}}
        \kern 0.5 cm
        \mbox{\hrefWithoutArrow{https://guswns1659.github.io}{{\footnotesize\faLink}\hspace*{0.13cm}https://guswns1659.github.io}}
        \kern 0.5 cm
        \mbox{\hrefWithoutArrow{https://github.com/guswns1659}{{\footnotesize\faGithub}\hspace*{0.13cm}https://github.com/guswns1659}}
        \kern 0.5 cm
    \end{header}

    \vspace{0.3 cm - 0.3 cm}


    \section{Who AM I}

    \begin{onecolentry}
        \begin{highlightsforbulletentries}

            \item 금융 도메인에서 요구하는 \textbf{높은 정합성을 유지}하기 위해 \textbf{동시성 제어 및 멱등성}을 보장하는 개발 역량을 가지고 있습니다.

            \item 2주 단위 스프린트 플래닝과 회고를 통해 \textbf{점진적 발전을 추구}하고 페어 프로그래밍 도입하여 \textbf{코드 품질 향상을 위해 노력}합니다.

        \end{highlightsforbulletentries}
    \end{onecolentry}

    \section{Experience}


    \begin{twocolentry}{
        2021.07 - 현재

        3년 2개월
    }
        \textbf{카카오페이}, {\footnotesize Backend Software Engineer}
        \begin{highlights}

            \item \textbf{\footnotesize 테크핀 도메인 개발 역량} : {\scriptsize 충전, 송금, 신한은행 등 외부사와 금융 제휴 상품 개발}
            \item \textbf{\footnotesize 무결성 시스템 개발 역량} : {\scriptsize 금액 무결성을 위한 데이터 검증 역량, 동시성 관리}
            \item \textbf{\footnotesize 협업 역량} : {\scriptsize 페어프로그래밍, 스프린트, 이벤트 스토밍}
        \end{highlights}
    \end{twocolentry}

    \vspace{0.2 cm}

    \begin{twocolentry}{
        2020.10 - 2021.07

        9개월
    }
        \textbf{뱅크샐러드}, {\footnotesize Backend Software Engineer}
        \begin{highlights}

            \item {\footnotesize 금융사와 PLCC 카드 개발, Open API 통신 개발}
        \end{highlights}
    \end{twocolentry}


    \section{Projects}


    \begin{twocolentry}{
        2024.02 - 2024.08

        6개월
    }
        \textbf{실시간 택스 리펀 프로젝트}
        \begin{highlights}
        {\footnotesize
            \item \textbf{설명} : 카카오페이 통해서 즉시 부가세를 환급받는 프로젝트
            \item \textbf{역할} :
            \begin{highlights}
                \item 정확한 정산을 위한 데이터 검증 작업 및 중복 호출 대비한 멱등적 API 설계
                \item RSA + SHA-256을 이용한 서명으로 데이터 무결성 검증
                \item 해외 금융사 개발 담당자와 API 개발 및 네트워크 통신을 위한 커뮤니케이션 담당
            \end{highlights}
        }
        \end{highlights}
    \end{twocolentry}

    \vspace{0.2 cm}

    \begin{twocolentry}{
        2023.07 - 2024.01

        5개월
    }
        \textbf{결제 반응형 적금 프로젝트}
        \begin{highlights}
        {\footnotesize

            \item \textbf{설명} : 결제한 금액의 일부를 적금으로 모아주는 프로젝트
            \item \textbf{성과} : 적금 계좌 \textbf{30,000건} 이상 개설
            \item \textbf{역할} :
            \begin{highlights}
                \item 분산 트랜잭션으로 나눠진 송금 기능을 안정적으로 개발하여 \textbf{송금 300만건} 정상 처리
                \item 은행 지연 대비 서킷브레이커 적용
                \item PM과 개발자의 동일한 컨텍스트 공유를 위한 이벤트 스토밍 및 스프린트 문화 도입
            \end{highlights}
        }
        \end{highlights}
    \end{twocolentry}

    \vspace{0.2 cm}

    \begin{twocolentry}{
        2022.10 - 2023.01

        3개월
    }
        \textbf{마이데이터 계좌 연결 프로젝트}
        \begin{highlights}
        {\footnotesize

            \item \textbf{설명} : 기존 1건씩 가능했던 계좌 연결 기능을 다건 연결로 개선하는 프로젝트
            \item \textbf{성과} : 직전 월 대비 평균 계좌 연결 건수 \textbf{85\% 증가} (54만건 -> 100만건)
            \item \textbf{역할} :
            \begin{highlights}
                \item 비동기 동시 호출을 적용하여 일반 동기 호출 대비 레이턴시 3초 -> 0.5초로 개선
                \item 폴링 및 카프카 이벤트 발행 방식으로 높은 처리량과 빠른 응답 보장
            \end{highlights}
        }
        \end{highlights}
    \end{twocolentry}

    \vspace{0.2 cm}

    \begin{twocolentry}{
        2022.07 - 2022.11

        4개월
    }
        \textbf{마이데이터 계좌 송금 프로젝트}
        \begin{highlights}

        {\footnotesize
            \item \textbf{설명} : 연결된 마이데이터 계좌에서 타 계좌로 송금할 수 있는 기능 제공하는 프로젝트
            \item \textbf{성과} : 분기 대비 송금 일 건수 \textbf{28\% 증가} (180만 건 -> 230만건)
            \item \textbf{역할} :
            \begin{highlights}
                \item 잔액 정합성을 위해 레디스를 이용한 분산락을 사용하여 동시성 관리
                \item 타임아웃 등으로 데이터 정합성 불일치 건들 보정 프로세스 개발
                \item 코드리뷰 비용 감소와 동일한 컨텍스트 공유를 위한 페어프로그래밍 진행
            \end{highlights}
        }
        \end{highlights}
    \end{twocolentry}

    \vspace{0.2 cm}

    \section{Trouble Shooting}




    \begin{highlights}

        \item \href{https://tinyurl.com/yh36sr5p}{HibernateCursorItemReader로 대용량 데이터 처리 시 OOME 발생 이유와 해결방법}
        \item \href{https://tinyurl.com/5by445vb}{스프링부트 부팅 시 성능 저하와 Warm-up}
        \item \href{https://tinyurl.com/9yjcsauf}{Spring Data JPA 페이징 시 OFFSET으로 인한 성능 저하와 해결법}

    \end{highlights}

    \vspace{0.2 cm}

    \section{Work-related Side Projects}

    \begin{highlights}

        \item \textbf{코드리뷰 지연 방지를 위한 PR 알림 봇} : {\footnotesize MSA로 프로젝트가 많아지면서 코드 리뷰의 불편함을 해소하기 위한 프로젝트}
            \begin{highlights}
                {\footnotesize
                    \item \textbf{역할} : \href{https://docs.github.com/en/rest/pulls/pulls?apiVersion=2022-11-28\#list-pull-requests}{Gihub Pull Request API} 연동하여 리뷰가 필요한 pr 조회하고 슬랙으로 매일 아침 알림
                    \item \textbf{개선} : 일 평균 20건 -> 7건 감소
                }
            \end{highlights}
        \item \textbf{배포 브랜치 자동 관리 시스템} : {\footnotesize 휴먼에러를 방지하기 위해 배포 브랜치를 자동으로 생성해주는 프로젝트}
            \begin{highlights}
                {\footnotesize
                    \item \textbf{역할} : \href{https://docs.github.com/ko/webhooks/webhook-events-and-payloads\#pull_request_review_comment}{Github hook} 연동하여 main P.R이 approved 받으면 자동으로 staging P.R을 자동으로 생성
                    \item \textbf{개선} : 수동으로 P.R 만들어야하는 불편함 해소, 안전한 main 브랜치 관리에 기여
                }
            \end{highlights}

    \end{highlights}


    \section{Education}


    \begin{twocolentry}{
        2022
    }
        \begin{highlights}

            \item \textbf{넥스트 스텝}, {\footnotesize \href{https://edu.nextstep.camp/c/VI4PhjPA}{인프라 공방} 수강}

        \end{highlights}
    \end{twocolentry}

    \vspace{0.2 cm}


    \begin{twocolentry}{
        2022
    }
        \begin{highlights}

            \item \textbf{넥스트 스텝}, {\footnotesize \href{https://edu.nextstep.camp/c/8fWRxNWU}{TDD, 클린 코드 with Java} 수강}

        \end{highlights}
    \end{twocolentry}

    \vspace{0.2 cm}

    \begin{twocolentry}{
        2020
    }
        \begin{highlights}

            \item \textbf{코드스쿼드 부트캠프}, {\footnotesize 백엔드 과정 수료}

        \end{highlights}
    \end{twocolentry}

    \vspace{0.2 cm}

    \begin{twocolentry}{
        2012 - 2019
    }
        \begin{highlights}

            \item \textbf{경희대학교}, {\footnotesize 호텔경영학과}

        \end{highlights}
    \end{twocolentry}


    \section{Specical Experience \& Certificate}

    \begin{twocolentry}{
        2024
    }
        \begin{highlights}

            \item \textbf{카카오페이 기술블로그 2건 발행},
            \begin{highlights}
                \footnotesize{
                    \item \href{https://tech.kakaopay.com/post/jack-k8s-internals-part-1/}{쓰기만 했던 개발자가 궁금해서 찾아본 쿠버네티스 내부 1편}
                    \item \href{https://tech.kakaopay.com/post/jack-k8s-internals-part-2/}{쓰기만 했던 개발자가 궁금해서 찾아본 쿠버네티스 내부 2편}
                }
            \end{highlights}

        \end{highlights}
    \end{twocolentry}

    \vspace{0.2 cm}


    \begin{twocolentry}{
        2024
    }
        \begin{highlights}

            \item SQLD

        \end{highlights}
    \end{twocolentry}

    \vspace{0.2 cm}


    \begin{twocolentry}{
        2022
    }
        \begin{highlights}

            \item \textbf{코드스쿼드 부트캠프 리뷰어}, {\footnotesize \href{https://tinyurl.com/4kzx9m9d}{수강생 프로젝트 코드리뷰 지원}}

        \end{highlights}
    \end{twocolentry}

    \vspace{0.2 cm}

    \begin{twocolentry}{
        2020
    }
        \begin{highlights}

            \item \textbf{오픈소스 컨트리뷰톤}, {\footnotesize \href{https://launchpad.net/~ubuntu-l10n-ko}{우분투 한국어 번역 프로젝트 참여}}

        \end{highlights}
    \end{twocolentry}


    \section{Technologies}


    \begin{highlights}
        \item \textbf{Languages:} Java, Kotlin
    \end{highlights}

    \vspace{0.2 cm}

    \begin{highlights}
        \item \textbf{Framework \& Library:} Spring Boot, Spring Batch, Spring Data Jpa, Spring AOP, Gradle, resilience4j
    \end{highlights}

    \vspace{0.2 cm}

    \begin{highlights}
        \item \textbf{Infra: } Kubernetes, Mysql, Redis, Kafka
    \end{highlights}


\end{document}